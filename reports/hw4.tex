\documentclass[10pt]{article}
\usepackage{latexsym}
\usepackage{natbib}
\usepackage{graphicx}
\usepackage{subfigure}
\usepackage{listings}
\usepackage{algorithm}
\usepackage{algpseudocode}
\usepackage{booktabs}
\usepackage{multirow}
\usepackage{siunitx}

\title{Homework 4: Project-Related Paper Report}
\author{Yu Feng}
\date{4/20/2015}

\begin{document}
\maketitle

\section{Summary}\label{sec:intro}
Using natural language to write programs is a challenge problem.
Le~\cite{smart} addressed the specific problem of synthesizing smartphone 
automation scripts from natural language. When the user fires a task in 
natural language like this:
\begin{quotation}
``When I receive a new SMS and I am driving, reply the send ``I'm driving."."
\end{quotation}
his system will generate the following executable script:
\small\begin{verbatim}
when (number, content) := MessageReceived()
    if (IsConnectedToBTDevice(Car_BT) then
        SendMessage(number, "I'm driving");
\end{verbatim}
Specifically, Le's approach has two key ingredients: First, design
a domain specific language to bridge the gap between natural language and the
target script; Second, for the data- flow that can not be inferred by 
standard NLP techniques, he uses techniques from the Program Synthesis community
to infer the missing relations.

The paper reduces the task of generating smartphone script to weaving a set of 
related APIs. To achieve this goal, his system first identify a set of components 
through standard NLP techniques like parse tree and bag-of-words as features; 
To infer the relations among components, it adopts the rule-based technique to 
generate the relations and uses program synthesis to infer the missing ones. For
the relation that has multiple candidates, it designs a ranking scheme to pick up
the one with a highest probability.

The paper evaluates the technique through 50 online task descriptions and the result
sounds promising: the precision is over 90\%.


\section{Improvement}\label{sec:alg}
There are several improvements we can make based on the disadvantages of the paper:

First, even though the paper claims that one of its contributions is to combine the
advantage of both NLP and program synthesis, it actually fails to leverage the 
advantage of modern NLP. For instance, it heavily relies on a set of so-called 
``expert-rules" to parse the natural language. It completely
\subsection{Evaluation}


\subsection{Conclusion}


\bibliographystyle{unsrt}
\bibliography{hw4}
\end{document}
